% Options for packages loaded elsewhere
\PassOptionsToPackage{unicode}{hyperref}
\PassOptionsToPackage{hyphens}{url}
\PassOptionsToPackage{dvipsnames,svgnames,x11names}{xcolor}
%
\documentclass[
]{article}
\usepackage{amsmath,amssymb}
\usepackage{lmodern}
\usepackage{iftex}
\ifPDFTeX
  \usepackage[T1]{fontenc}
  \usepackage[utf8]{inputenc}
  \usepackage{textcomp} % provide euro and other symbols
\else % if luatex or xetex
  \usepackage{unicode-math}
  \defaultfontfeatures{Scale=MatchLowercase}
  \defaultfontfeatures[\rmfamily]{Ligatures=TeX,Scale=1}
\fi
% Use upquote if available, for straight quotes in verbatim environments
\IfFileExists{upquote.sty}{\usepackage{upquote}}{}
\IfFileExists{microtype.sty}{% use microtype if available
  \usepackage[]{microtype}
  \UseMicrotypeSet[protrusion]{basicmath} % disable protrusion for tt fonts
}{}
\makeatletter
\@ifundefined{KOMAClassName}{% if non-KOMA class
  \IfFileExists{parskip.sty}{%
    \usepackage{parskip}
  }{% else
    \setlength{\parindent}{0pt}
    \setlength{\parskip}{6pt plus 2pt minus 1pt}}
}{% if KOMA class
  \KOMAoptions{parskip=half}}
\makeatother
\usepackage{xcolor}
\usepackage[margin=1in]{geometry}
\usepackage{graphicx}
\makeatletter
\def\maxwidth{\ifdim\Gin@nat@width>\linewidth\linewidth\else\Gin@nat@width\fi}
\def\maxheight{\ifdim\Gin@nat@height>\textheight\textheight\else\Gin@nat@height\fi}
\makeatother
% Scale images if necessary, so that they will not overflow the page
% margins by default, and it is still possible to overwrite the defaults
% using explicit options in \includegraphics[width, height, ...]{}
\setkeys{Gin}{width=\maxwidth,height=\maxheight,keepaspectratio}
% Set default figure placement to htbp
\makeatletter
\def\fps@figure{htbp}
\makeatother
\setlength{\emergencystretch}{3em} % prevent overfull lines
\providecommand{\tightlist}{%
  \setlength{\itemsep}{0pt}\setlength{\parskip}{0pt}}
\setcounter{secnumdepth}{-\maxdimen} % remove section numbering
\usepackage{caption}
\captionsetup[figure]{labelformat=empty}
\ifLuaTeX
  \usepackage{selnolig}  % disable illegal ligatures
\fi
\IfFileExists{bookmark.sty}{\usepackage{bookmark}}{\usepackage{hyperref}}
\IfFileExists{xurl.sty}{\usepackage{xurl}}{} % add URL line breaks if available
\urlstyle{same} % disable monospaced font for URLs
\hypersetup{
  pdftitle={ESS 575: Final Project},
  pdfauthor={George Woolsey},
  colorlinks=true,
  linkcolor={blue},
  filecolor={Maroon},
  citecolor={Blue},
  urlcolor={Blue},
  pdfcreator={LaTeX via pandoc}}

\title{ESS 575: Final Project}
\author{George Woolsey}
\date{10 December, 2022}

\begin{document}
\maketitle

\hypertarget{introduction}{%
\section{Introduction}\label{introduction}}

In this particular example we ask the question: How does variation in
weather modify feedbacks between population density and population
growth rate in a population of large herbivores occupying a landscape
where precipitation is variable in time? Answering this question
requires a model that portrays density dependence, effects of
precipitation, and their interaction.

Write a brief introduction to the problem you are studying. The first
few sentences should provide the broad context --why is the general
topic important to the discipline of ecology, citing a few key papers.
Proceed to explain why your specific work will advance understanding of
the broad topic. Describe the core questions and or objectives of the
work. The introduction should resemble a funnel -- big topics at the top
narrowing to specific questions at the bottom. You want to convince the
reader that your specific model system is well poised to provide general
insight in ecology. By the way, this is how all papers and proposals
should begin.

\begin{figure}

{\centering \includegraphics[width=1\linewidth,height=1\textheight]{../data/WFF_map} 

}

\end{figure}

\textbf{Fig. 1.} 2013 West Fork Fire Complex on the Rio Grande National
Forest (Colorado, USA) courtesy of (Verdin, Dupree, \& Stevens, 2013)

\textbf{Fig. 2.} Management treatment and natural disturbance
interaction matrix representing the 6 group-level classifications () in
the Bayesian dynamic hierarchical model utilized in this analysis

\hypertarget{data-and-research-design}{%
\section{Data and Research design}\label{data-and-research-design}}

Describe the data that you will use to fit models and how they were
collected. Imagine that you were giving me one of the word problems on
writing hierarchical models I gave you (payback). I need to know enough
to be able to write the model myself. I don't need to know details of
methods for data collection, but I do need to understand the design.
Explain spatial and or temporal structure of the data and describe
sources of calibration and sampling uncertainty

\hypertarget{model}{%
\section{Model}\label{model}}

The modelling objective in this analysis is to fit a logistic growth
model for forest regrowth using the Landsat Net Primary Production (NPP)
data product while accounting for variance in modelling the ecological
process unrelated to forest regrowth. A Bayesian dynamic hierarchical
model (i.e., a ``state-space'' model) was developed to obtain posterior
distributions of the latent state and parameters of interest. The
hierarchical model implemented and described below includes a model of
the ecological process (unobserved latent state), a model linking the
process to observed data, and models for parameters. The final model
predicts forest NPP regrowth at the pixel level (\emph{i}) for each of
the 6 possible treatment, disturbance interactions (\emph{j}), for each
year (\emph{t}) over a 9-year regrowth period.

\textbf{Fig. 3.} Hierarchical Bayesian model of the dynamics of forest
productivity (NPP) regrowth on the Rio Grande National Forest (Colorado,
USA) following the 2013 West Fork Fire Complex. The true NPP for 30-m
pixel \emph{i} in treatment, disturbance interaction group \emph{j} at
time \emph{t} is modeled using the deterministic model (Eqn. 1), which
represents the effects of the true, unobserved NPP (\(z_{ijt}\)); the
climatic water deficit (\(x_{ijt}\)); and their interaction on forest
regrowth. See below for interpretation of the \(\boldsymbol{\beta}\)
parameters. The data model relates observed values of NPP (\(y_{ijt}\))
at pixel \emph{i} in treatment, disturbance interaction group \emph{j}
at time \emph{t} to the latent state (\(z\)). The solid lines show
stochastic relationships while the dashed lines show deterministic
relationships, implying that the quantities at the tails of the arrows
are known without error (Hobbs \& Hooten, 2015).

\hypertarget{deterministic-model}{%
\subsection{Deterministic model:}\label{deterministic-model}}

The logistic growth model was implemented as the ecological process
model of forest regrowth. Several studies have shown that the typical
trajectory of forest NPP overtime following stand-replacing disturbance
is represented by a rapid initial increase with a modest decline
thereafter (Gower et al., 1996; Law et al., 2003; Pregitzer \&
Euskirchen, 2004; Goulden et al., 2011). The logistic growth model has
been applied widely to forest growth dynamics including models of forest
succession and dispersion (Acevedo et al., 2012; Richit et al., 2019).

Forest regrowth rates are not constant over time as they depend on
existing forest cover, that is, they are density dependent. The logistic
growth model allows us to predict next year's forest productivity (NPP)
but it is always dependent on the previous year's productivity. In the
beginning (under zero forest cover conditions), forest regrowth is
nearly exponential, with increases close to the maximum instantaneous
growth rate \(r_{max}\). There is a constant linear decrease in the
instantaneous growth rate (\(r\)) as forest cover increases. Forest
growth eventually plateaus and fluctuates around the carrying capacity
(\(K\)) which represents the maximum forest cover at which the
instantaneous growth rate (\(r\)) is 0. The strength of density
dependence is represented by \(\frac{r_{max}}{K}\), where a negative
ratio would indicate that the growth rate decreases with increasing
forest cover.

The deterministic model used to represent the ecological process is a
special form of the logistic growth model, the Ricker equation, which
accounts for density-dependent growth:

\begin{align*}
z_{ijt} = g(\boldsymbol{\beta}_j, z_{ijt-1}, x_{ijt}) &= z_{ijt-1}e^{(\beta_{0j} \,+\, \beta_{1j} \cdot z_{ijt-1} \,+\, \beta_{2j}  \cdot x_{ijt} \,+\, \beta_{3j}  \cdot z_{ijt-1}  \cdot x_{ijt})}  \\
x &= \textrm{ climatic water deficit}
\end{align*}

The process model of the unobserved, true NPP (\(z_{ijt}\)) is indexed
by the subscripts:

\begin{itemize}
\tightlist
\item
  \emph{i} representing annual observations of NPP at the 30-m pixel
  level
\item
  \emph{j} representing each of the 6 possible treatment, disturbance
  interactions (Fig. 2)
\item
  \emph{t} denoting each of the 9 years spanning the regrowth period
  2014 to 2022
\end{itemize}

The biological interpretation of the \(\boldsymbol{\beta}\) parameters
are:

\begin{itemize}
\item
  \(\beta_{0j}\), the intercept, is analogous to the intrinsic, maximum
  forest growth rate (\(r_{max}\)) when forest cover is 0 and climatic
  water deficit is average for treatment, disturbance interaction
  \emph{j}
\item
  \(\beta_{1j}\) slope represents the magnitude of forest competition
  (i.e., the strength of density dependence) for treatment, disturbance
  interaction \emph{j}. In the Stochastic Ricker (logistic) Model
  \(\beta_{1j} = \frac{r_{max}}{K}\), where \(K\) is the carrying
  capacity
\item
  \(\beta_{2j}\) slope is a measure of the strength of the effect of
  variation in climatic water deficit for treatment, disturbance
  interaction \emph{j}
\item
  \(\beta_{3j}\) slope represents the magnitude of the effect of
  climatic water deficit on the effect of density for treatment,
  disturbance interaction \emph{j}
\end{itemize}

See Hobbs \& Hooten (2015, p.11) for a more thorough description of this
model for representing density-dependent growth.

\hypertarget{process-model}{%
\subsection{Process model}\label{process-model}}

The modelling objective is to fit a logistic growth model for forest
regrowth using remotely-sensed, modeled NPP data while accounting for
variance in modelling the ecological process unrelated to forest
regrowth. The deterministic model of the unobserved latent state
variable \(z\), which represents ``true'' NPP, is an imperfect
representation of the ecological process of forest regrowth and is
subject to process error. The process variance (\(\sigma^{2}_{p}\)) in
the deterministic model accounts for the failure of the model to
represent all the influences on the true state.

The process model uses a lognormal distribution for the latent state
(\(z\)) to represent the strictly non-negative, true value of NPP:

\[
z_{ijt} \sim {\sf lognormal} \biggl(\log \bigl( g(\boldsymbol{\beta}_j, z_{ijt-1}, x_{ijt})  \bigr)  , \sigma^{2}_{p} \biggr)
\]

\hypertarget{data-model}{%
\subsection{Data model}\label{data-model}}

The data model relates observed values of NPP (\(y_{ijt}\)) at pixel
\emph{i} in treatment, disturbance interaction group \emph{j} at time
\emph{t} to the latent state \(z_{ijt}\):

\[
y_{ijt} \sim {\sf lognormal} \bigl(z_{ijt}, \sigma^{2}_{d} \bigr)
\]

To represent sources of noise related to measurement error (e.g.~failure
to perfectly observe NPP from optical satellite remote sensing;
imperfections of the algorithm used to model NPP), observed NPP is
modeled as draws from a lognormal distribution with \(z\) as the mean
value and a variance term \(\sigma^{2}_{d}\). The magnitude of the
observation uncertainty (\(\sigma^{2}_{p}\)) represents measurement
error. Future work building on the dynamic hierarchical model presented
here could integrate analysis of the relationship between fine-scale
flux tower measurements and satellite-based estimates of NPP (e.g.~Jay
et al., 2016) to relate the unobserved, true NPP to observed NPP.

\hypertarget{full-model}{%
\subsection{Full Model}\label{full-model}}

The full model, including prior distributions, is specified by the
following statement in which items in bold represent matrices:

\begin{align*}
\bigl[ \boldsymbol{z},\boldsymbol{\beta},\boldsymbol{\mu_{\beta}}, \sigma^{2}_{p}, \sigma^{2}_{d} \mid \boldsymbol{y} \bigr] &\propto \\
&\prod_{i=1}^{n}\prod_{j=1}^{6}\prod_{t=2}^{9} {\sf lognormal} \bigl(y_{ijt} \mid z_{ijt}, \sigma^{2}_{d} \bigr) \\ 
&\times \; {\sf lognormal} \biggl(z_{ijt} \biggm| \log \bigl( g(\boldsymbol{\beta}_j, z_{ijt-1}, x_{ijt})  \bigr)  , \sigma^{2}_{p} \biggr)\\
&\times \; {\sf normal} \bigr( z_{ij1} \mid y_{ij1} \bigr) \\
&\times \; {\sf uniform} \bigr( \sigma^{2}_{p} \mid 0, 1 \bigr) \\
&\times \; {\sf uniform} \bigr( \sigma^{2}_{d} \mid 0, 1 \bigr) \\
&\times \; \text{multivariate normal}\left(\left(\begin{array}{c}
  \beta_{0j}\\
  \beta_{1j}\\
  \beta_{2j}\\
  \beta_{3j}
\end{array}\right) \bigg|
\left(\begin{array}{c}
  \mu_{\beta_{0}}\\
  \mu_{\beta_{1}}\\
  \mu_{\beta_{2}}\\
  \mu_{\beta_{3}}
\end{array}\right)
,\boldsymbol{\Sigma}\right) \\
&\times \; {\sf Wishart} \left(\boldsymbol{\Sigma} \, \bigg|
\left(\begin{array}{c}
  1 & 0 & 0  & 0 \\
  0 & 1 & 0  & 0 \\
  0 & 0 & 1  & 0 \\
  0 & 0 & 0  & 1
\end{array}\right)
,4+1\right) \\
&\times \, \prod_{k=0}^{3} {\sf normal} \bigr( \mu_{\beta_{k}} \mid 0,100000 \bigr)
\end{align*}

Our hypothesis is that different treatment and disturbance interaction
groups had different levels of resistance to the fire disturbance
(represented by the model intercept) and also different post-fire
recovery patterns (represented by the slope of the model). To test this
hypothesis, we model group (subscript \emph{j}) effects on intercepts
and slopes. In order to understand group effects on multiple parameters
(\(\beta_{j0}, \beta_{j1},\beta_{j2},\beta_{j3}\) in our model), we
account for the way that the parameters covary using a scaled
inverse-Wishart model. See Gelman \& Hill (2009, pg. 376) for details.
The covariance matrix \(\boldsymbol{\Sigma}\) (i.e., variance covariance
matrix) is an \$m \times m \$ matrix with ones on the diagonal and zeros
on the off diagonals where \(m\) is the number of coefficients including
the intercept. Normal priors on the model coefficients
(\(\boldsymbol{\beta}\)) were uninformative.

Uniformly distributed vague priors were utilized for the process error
(\(\sigma^{2}_{p}\)) and the measurement error (\(\sigma^{2}_{d}\)).

\hypertarget{analysis}{%
\section{Analysis}\label{analysis}}

Write a section that describes the computational procedures you will
use, tests for convergence, and posterior predictive checks. Describe
any important derived quantities. This should read the like the section
you would write for a paper to be submitted to a journal. Consult
published Bayesian papers for examples.

\newpage

\hypertarget{references}{%
\section{References}\label{references}}

Gelman, A., \& Hill, J. (2009). Data analysis using regression and
multilevel/hierarchical models. Cambridge university press.

Jay, S., Potter, C., Crabtree, R., Genovese, V., Weiss, D. J., \& Kraft,
M. (2016). Evaluation of modelled net primary production using MODIS and
landsat satellite data fusion. Carbon Balance and Management, 11(1),
1-13.

Verdin, K. L., Dupree, J. A., \& Stevens, M. R. (2013). Postwildfire
debris-flow hazard assessment of the area burned by the 2013 West Fork
Fire Complex, southwestern Colorado. US Department of the Interior, US
Geological Survey.

\end{document}
